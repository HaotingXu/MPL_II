\documentclass[10pt,a4paper]{ctexart}
\input{macros.tex}
\geometry{a4paper,left=2cm,right=2cm,top=2cm,bottom=2cm}
\graphicspath{{figure/}}
\usepackage{enumerate}
\usepackage{listings}
%\usepackage{fancyhdr}
%\cpic{<尺寸>}{<文件名>}}用于生成居中的图片。
\newcommand{\cpic}[2]{
\begin{center}
\includegraphics[scale=#1]{#2}
\end{center}
}
%\cpicn{<尺寸>}{<文件名>}{<注释>}用于生成居中且带有注释的图片,其label为图片名。
\newcommand{\cpicn}[3]
{
\begin{figure}[H]
\cpic{#1}{#2}
\caption{\color{red}#3\label{#2}}
\end{figure}
}

\crefname{equation}{}{}
\crefname{figure}{图}{图}
\crefname{footnote}{注释}{注释}
\crefname{table}{表}{表}
\title{实验D2 外腔式半导体激光器(ECDL)的相关实验(第一次实验)}
\date{}
\begin{document}
\maketitle
\begin{center}
\begin{tabular}{|c|c|c|c|}
\hline
		实验方案 &实验记录  &分析讨论 &总成绩\\
		\hline
		        &          &          &  \\
	    \hline
	\hline 
	年级、专业: &17级物理学 &组号:& 3 \\
	\hline
	姓名:& 徐昊霆 &学号:&17353071  \\
	\hline
	日期:& \today &教师签名: &  \\
    \hline	
\end{tabular}
\end{center}
    \begin{enumerate}
 \item 实验报告由三部分组成:
 \begin{enumerate}
  \item[1)]预习报告:(提前一周)认真研读\textbf{\uline{实验讲义}},弄清实验原理;实验所需的仪器设备、用具及其使用(强烈建议到实验室预习),完成讲义中的预习思考题;了解实验需要测量的物理量,并根据要求提前准备实验记录表格(由学生自己在实验前设计好,可以打印)。预习成绩低于50\%者不能做实验{\color{red} (实验D2和D3需要提前一周的周四完成预习报告交任课老师批改,批改通过后,才允许做实验)}。

  \item[2)]实验记录:认真、客观记录实验条件、实验过程中的现象以及数据。实验记录请用珠笔或者钢笔书写并签名({\color{red}用铅笔记录的被认为无效})。{\color{red}保持原始记录,包括写错删除部分,如因误记需要修改记录,必须按规范修改。}(不得输入电脑打印,但可扫描手记后打印扫描件);离开前请实验教师检查记录并签名。
  \item[3)]分析讨论:处理实验原始数据(学习仪器使用类型的实验除外),对数据的可靠性和合理性进行分析;按规范呈现数据和结果(图、表),包括数据、图表按顺序编号及其引用;分析物理现象(含回答实验思考题,写出问题思考过程,必要时按规范引用数据);最后得出结论。
 \end{enumerate}
 \textbf{实验报告}就是预习报告、实验记录、和数据处理与分析合起来,加上本页封面。
 \item 每次完成实验后的一周内交\textbf{实验报告}。
 \item 除实验记录外,实验报告其他部分建议双面打印。
\end{enumerate}

    
\newpage
\tableofcontents
\newpage
\section{实验原理与方案}
\subsection{实验目的}
\begin{enumerate}
 \item[1.] 熟悉法布里-珀罗干涉仪的工作原理、结构、特点、调节和使用方法。
  \item[2.] 掌握应用法布里-珀罗干涉仪测量 ECDL 的频率输出模式。
\item[3.] 熟悉光栅波长计的工作原理,应用波长计测量 ECDL 的频率输出模式。
\item[4.] 熟悉锂原子的多普勒吸收,并运用锂原子的吸收谱线校准波长计。
\item[5.] 掌握在锂原子吸收谱线附近实现最大的无跳模范围的调节。
\end{enumerate}


\subsection{仪器用具}
\begin{table}[H]
  \caption{实验用具}
\cpic{0.4}{t1}
\end{table}

\subsection{实验安全注意事项}
\begin{enumerate}
\item[1.] 警告:使用的激光器输出功率达到了 class 3 级,能够对人眼带来永久的损伤,严禁
任何激光直射入眼睛!
!试验过程中需要佩戴保护眼镜。
\item[2.] 警告:在激光器开启后,严禁眼睛与激光平台同一个高度,防止激光射入眼睛!严
禁坐着,务必避免与激光在同一高度!
\item[3.] 警告:实验过程中严禁带任何手表、首饰等,防止激光反射进入眼睛。
\item[4.] 注意事项:不能用手触摸反射镜。调节镜座两个螺丝时必须配合一起调,不可只调
一个螺丝,不可将某一个螺丝拧的过紧,也不可将螺丝拧的过松。在摆置光路之前,建议把螺丝的行程调节在中间,方便实验过程中调节。实验中需尽量避免用手直接触镜片的光学面。若不小心触摸了光学表面,需尽快用镜头纸或擦镜布擦拭干净。
\item[5.]实验结束后将实验器件按照顺序放回原件盒。
\end{enumerate}
\subsection{实验原理}
外腔式半导体激光器(ECDL)就是为了获得频宽很窄的激光。如何获得频宽很窄的激光呢?一共分三步:第一步,产生激光;第二步,用光学谐振腔和光栅进行选频;第三步算算中间过程的增益,让想要的频率光强强一点。
\subsubsection{谐振腔}
谐振腔的原理和我们光学学过的法布里-珀罗干涉仪的原理相同,在实验中,我们也要用到法布里-珀罗干涉仪来探测激光器发出的是否是单模激光。在光轴的方向向FP干涉仪入射一束光,经过很多次反射,可以推导出出射的光强为
\beq
I = \frac{I_0}{1+F\sin^2(\delta/2)}
\eeq
其中$F = 4R/(1-R)^2$被叫做精细度,其中$R$是反射率。当频率和两个反射镜之间的距离合适,光程差$\delta$将是$2\pi$的整数倍,从而出射光达到最大。

两个极大值对应的频率差定义为FP谐振腔的{\color{red} 自由光谱范围},自由光谱范围由公式$v_{\rm FSR}=c/4L$给出(这里的系数$4$是因为法布里珀罗干涉仪的两块反射镜不是平的)。如果进行仔细地对准,使得入射光沿着FP腔的光轴入射,并对入射光进行空间模式匹配,那么就可以接近完美地消除其他高阶模式的光,从而出射光强和 modes 的曲线变成一条条竖线。

\subsubsection{ECDL}
半导体激光器先产生激光,先经过内腔选择模式和出射方向(腔长$1$mm左右,Internal mode),再入射到光栅上衍射进行选模(Grating),再入射到外腔进行选模(腔长远大于内腔,所以筛选频率间距远小于内腔,External Mode)。在多个频率限制的激光器中,最后输出的光强还要乘上增益。$\text{总增益}=\text{激光二极管增益}\times \text{二极管内腔选频}\times \text{光栅增益}\times\text{外腔选频}$。最后的效果达到让想要的频率的光强远远大于其他频率的光强。选频的示意图如图~\ref{ECDL}所示。
\cpicn{0.3}{ECDL}{选频示意图}
\subsubsection{如何调节}
\begin{enumerate}
	\item 光栅选频(Grating Profile)由光栅方程决定,通过调节光栅的角度可实现grating profile频率的移动。这个角度又可以通过PTZ\footnote{something we don't understand}的电压来调节。
	\item 激光器出射的强度(Medium Gain)由激光二极管的增益介质决定,它的中心位置可通过{\color{red} 温度}来调节,它的强度可以通过{\color{red} 电流}来调节。
	\item Internal mode 的改变是通过腔长的改变实现的,而腔长的改变则通过电流带来的热胀冷缩效应。
	\item External mode 的改变也是通过光栅的角度、二极管的温度变化实现
\end{enumerate}
因此我们最后要实现单模运行或者最大无跳模范围,就需要调节激光二极管的电流、温度和光栅的角度。
\newpage
\section{实验步骤与记录}
\begin{center}
\begin{tabular}{|p{8em}|p{8em}|p{8em}|p{8em}|}
	\hline 
	专业:     &Physics       &年级:      & 17     \\
	\hline
	姓名:& 徐昊霆 &学号:&17353071  \\
	\hline
	室温:&                    &实验地点 & 教学楼 \\
	\hline	
	学生签名: &\includegraphics[scale=0.2]{sign} & 评分: & \\
	\hline
	日期: & \today & 教师签名:&  \\
	\hline
\end{tabular}
\end{center}
第一次实验内容概括:
\begin{enumerate}
	\item 打开激光器。
	\item 分别用FP干涉仪和波长计探测激光器输出的单模。
	\item 分别改变温度、PTZ的DC电压、激光器电流,探究对于激光器输出单模的影响。
\end{enumerate}

\subsection{ECDL激光器输出频率的单模实现}
\begin{enumerate}
\item  按照激光器使用说明书,先打开激光器电源,使激光器工作在适合的温度 20.0$^\circ {\rm C}$。
\item 设置激光器的 PZT 控制电压(DC)和扫描电压(AC)为 0V;
\item 打开激光器 steady 按钮,将光功率计置于激光器的前方,从 0mA 缓慢增加激光器的输入电流至激光输出功率为 3 mW 附近(电流一般为 20 mA 左右)。注意:电流范围 0-55mA,严禁超过电流使用激光器!
\item {\color{red}采用FP腔探测激光器输出的单模} 耦合激光进入 FP 腔,调节耦合系统,实现最佳的耦合。

{\color{blue} (来自说明书的对准步骤)将输入光圈关闭到最小,然后将
光束对准光圈的中心,最方便的方法是通过两个折叠镜将光束对准
到输入光圈上。保持后光圈完全打开,然后开始扫描。确保完整
扫描在示波器上可见,并显示来自检测器的信号;对于ignition alignment,调整示波器的灵敏度最大。

使用反光镜安装座的倾斜/倾斜调节,直到光束穿过SA200的主体居中,即直到
您开始在示波器上看到模式。将反光镜安装座调整至
将光束保持在设备的中央。光束居中后,可以使用
通过监视示波器上传输模式的形状和大小来控制两个输入镜。的
干涉仪将准备好进行测量。}

\item 在 FP 腔的输出信号上得到典型的单模输出波形,记录实验结果。在没有实现最佳的耦
合时会出现比 ${\rm TEM}_{q00}$ 和 $\text{TEM}_{q+1\ 00}$ 更高阶的模式。实验中需要记录最佳耦合的结果。并
表明哪些是$\text{TEM}_{q00}$和 $\text{TEM}_{q+1\ 00}$  模式。
\item  {\color{red} 采用波长计探测激光器输出的单模。} 耦合激光进入波长计的单模光纤,调节耦合系统,实现最佳的耦合。
备注:1. 这个步骤需要结合预习的内容,制定相关的实验步骤。2. 在耦合进入光纤的时候
最好保留 FP 腔的耦合不变,因为后面的实验中 FP 腔和波长计都必须同时用到。
\item 在波长计的输出信号上得到典型的单模输出波形,记录实验结果。
\item 实现激光器的单模输出后,{\color{red}研究激光器电流、PZT 的 DC 驱动电压对单模输出的影响},
记录数据。然后对数据进行简单的分析。备注:1. 推荐激光器的电流变化的步进值 0.5 mA,改变范围 10 mA 以内。2. PZT 的 DC 电压的步进值 1 V,改变范围 15 V 以内。3. 注意改变 PZT 的 DC 电压时,激光器的电流改
变,记录相关数值。分析为什么。
\item 在做第 8 步的时候需要同时记录波长计的读数,然后分析激光器的输出波长随激光器电
流、PZT 的 DC 驱动电压的变化规律。
\item 改变激光器的温度,记录波长计的读数变化。温度的改变步进值 0.1$^\circ {\rm C}$,范围 1$^\circ {\rm C}$。温度
的最小值 19$^\circ {\rm C}$,最大值 24$^\circ {\rm C}$,激光器的温度不能超过这个数值!最后分析激光器温度
对波长的影响。
\item 结合上面激光器电流、PZT 的 DC 电压以及温度对激光器输出频率模式、波长的影响,
分析应该如何实现激光输出频率的单模变化,以及如何在指定波长附近实现激光频率单模的调节。(这部分在分析与讨论中)。
\end{enumerate}
\begin{table}[H]
\begin{center}
    \caption{探究电流对于激光器单模输出的影响(请在旁边标注电压和温度)}
    \begin{tabular}{|c|l|l|c|l|l|}
\hline
测量次数 & 电流/mA & 波长计读数/nm &测量次数 & 电流/mA & 波长计读数/nm \\ \hline
1    &       &  &11    &       &          \\ \hline
2    &       &   &12    &       &         \\ \hline
3    &       &     &13    &       &       \\ \hline
4    &       &       &14    &       &     \\ \hline
5    &       &   &15    &       &         \\ \hline
6    &       &   &16    &       &         \\ \hline
7    &       &   &17    &       &         \\ \hline
8    &       &   &18    &       &         \\ \hline
9    &       &   &19    &       &         \\ \hline
10   &       &   &20    &       &         \\ \hline
\end{tabular}
\end{center}
\end{table}

\begin{table}[H]
\begin{center}
    \caption{探究温度对于激光器单模输出的影响(请在旁边标注电压和电流)}
    \begin{tabular}{|c|l|l|}
\hline
测量次数 & 温度/$^\circ {\rm C}$ & 波长计读数/nm \\ \hline
1    &       &          \\ \hline
2    &       &          \\ \hline
3    &       &          \\ \hline
4    &       &          \\ \hline
5    &       &          \\ \hline
6    &       &          \\ \hline
7    &       &          \\ \hline
8    &       &          \\ \hline
9    &       &          \\ \hline
10   &       &          \\ \hline
\end{tabular}
\end{center}
\end{table}

\begin{table}[H]
\begin{center}
    \caption{探究PTZ的DC电压对于激光器单模输出的影响(请在旁边标注温度和电流)}
    \begin{tabular}{|c|l|l|}
\hline
测量次数 & 电压/V & 波长计读数/nm \\ \hline
1    &       &          \\ \hline
2    &       &          \\ \hline
3    &       &          \\ \hline
4    &       &          \\ \hline
5    &       &          \\ \hline
6    &       &          \\ \hline
7    &       &          \\ \hline
8    &       &          \\ \hline
9    &       &          \\ \hline
10   &       &          \\ \hline
11   &       &          \\ \hline
12   &       &          \\ \hline
13   &       &          \\ \hline
14   &       &          \\ \hline
15   &       &          \\ \hline
\end{tabular}
\end{center}
\end{table}
\subsection{实验中遇到的问题记录}



\newpage
\section{分析与讨论}
\begin{center}
\begin{tabular}{|c|c|c|c|}
	\hline 
	专业:     &Physics       &年级:      & 17     \\
	\hline
	姓名:& 徐昊霆 &学号:&17353071  \\
	\hline
	日期&  \today              & &  \\
	\hline	
	评分 & & 教师签名 & \\
	\hline
\end{tabular}
\end{center}



%\bibliographystyle{siam}
%\bibliography{cites}
\end{document}
